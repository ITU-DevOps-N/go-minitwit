\section{Lessons Learned Perspective}
\label{learned-perspective}

% Describe the biggest issues, how you solved them, and which are major lessons learned with regards to:
\subsection{Evolution and Refactoring}
The main issues during the development phase and in general the evolution of the course were:
\begin{itemize}
    \item Pace of the course
    \item Learning Golang
    \item Revising architecture often
    \item Backtracking errors.
\end{itemize}

The first issue was the \textbf{pace of the course} and the number of technologies which we had to add to our tech stack was challenging and work heavy. This limited our ability to keep up with the course material in the beginning since we simultaneously had to \textbf{learn a new language} and web framework. 

During the implementation of new components and \textbf{revisiting the architecture}, we spent a remarkable amount of time designing our system to make sure it was built the right way. With the load generated by the Simulator API, \textbf{backtracking errors} and performing bug fixing was a long and continuous work. It was often necessary to investigate metrics, logs and Bugsnag reports to identify the issue. 

Initially, we only released new application versions on Docker Hub with each new image in our continuous deployment pipeline. However, the image did not capture the entire state of the repository for each release, only the source files for building and running the application itself. We fixed this late in the course by implementing a new GitHub action to automatically create a GitHub release whenever we pushed to main. This release automatically made a changelog of all the commits that went into main since the last release, and the release version number was also updated automatically. In hindsight, it should have been there from the beginning to comply more with the DevOps way of thinking with more frequent releases. 

\subsection{Operation}
We had a difficult time implementing both monitoring and logging, which resulted in a hard time handling issues with our application. Also, documentation was often made after implementation and sometimes not together with pushing into development/main. Ideally, we should have been doing implementation and documentation parallel and enforcing documentation before pushing into development or at least production. 

Another issue was related to the developer experience. We wanted to have a philosophy of 'keeping the machine clean', meaning that we did not want everyone to download specific dependencies that the system needed, such as the Go version and DBMS version. We wanted a seamless setup of the environment where everyone was using the same versions with Docker. In Go, you can test your code modifications without compiling them into an executable. However, the way our architecture was setup meant that we had to compile the application for each change. For development purposes, this is very inconvenient, since compiling takes a lot of time for even a simple change. 

\subsection{Maintenance}
We were very static in our approach to working with maintenance as we mostly only worked once or twice a week on developing/maintaining our system. Ideally, we should have been more flexible in our way of working with a larger focus on maintenance, hence, working less but more days. This was also reflected in the many errors generated by the simulator, which during a week would aggregate to a lot of issues not handled. Our attempt to accommodate for this was to meet during weekends to catch up, however, this approach was still reactive and not proactive. 

\subsection{Reflection and description of our DevOps style}
For this project the application itself was rather secondary as setting up the whole system infrastructure was the focus. This means we actually never got to the point where we were adding that much new functionality or visualization to the application, thus hard to comply with the DevOps way of working. We also focused on making development as easy as possible since being efficient in implementing new features was a major key to the success of this project.
